\documentclass[conference,compsoc]{IEEEtran}
 
\usepackage{datetime}

% *** CITATION PACKAGES ***
%
\ifCLASSOPTIONcompsoc
  % IEEE Computer Society needs nocompress option
  % requires cite.sty v4.0 or later (November 2003)
  \usepackage[nocompress]{cite}
\else
  % normal IEEE
  \usepackage{cite}
\fi
% cite.sty was written by Donald Arseneau
% V1.6 and later of IEEEtran pre-defines the format of the cite.sty package
% \cite{} output to follow that of the IEEE. Loading the cite package will
% result in citation numbers being automatically sorted and properly
% "compressed/ranged". e.g., [1], [9], [2], [7], [5], [6] without using
% cite.sty will become [1], [2], [5]--[7], [9] using cite.sty. cite.sty's
% \cite will automatically add leading space, if needed. Use cite.sty's
% noadjust option (cite.sty V3.8 and later) if you want to turn this off
% such as if a citation ever needs to be enclosed in parenthesis.
% cite.sty is already installed on most LaTeX systems. Be sure and use
% version 5.0 (2009-03-20) and later if using hyperref.sty.
% The latest version can be obtained at:
% http://www.ctan.org/pkg/cite
% The documentation is contained in the cite.sty file itself.
%
% Note that some packages require special options to format as the Computer
% Society requires. In particular, Computer Society  papers do not use
% compressed citation ranges as is done in typical IEEE papers
% (e.g., [1]-[4]). Instead, they list every citation separately in order
% (e.g., [1], [2], [3], [4]). To get the latter we need to load the cite
% package with the nocompress option which is supported by cite.sty v4.0
% and later.

% *** GRAPHICS RELATED PACKAGES ***
%
\ifCLASSINFOpdf
  % \usepackage[pdftex]{graphicx}
  % declare the path(s) where your graphic files are
  % \graphicspath{{../pdf/}{../jpeg/}}
  % and their extensions so you won't have to specify these with
  % every instance of \includegraphics
  % \DeclareGraphicsExtensions{.pdf,.jpeg,.png}
\else
  % or other class option (dvipsone, dvipdf, if not using dvips). graphicx
  % will default to the driver specified in the system graphics.cfg if no
  % driver is specified.
  % \usepackage[dvips]{graphicx}
  % declare the path(s) where your graphic files are
  % \graphicspath{{../eps/}}
  % and their extensions so you won't have to specify these with
  % every instance of \includegraphics
  % \DeclareGraphicsExtensions{.eps}
\fi
% graphicx was written by David Carlisle and Sebastian Rahtz. It is
% required if you want graphics, photos, etc. graphicx.sty is already
% installed on most LaTeX systems. The latest version and documentation
% can be obtained at: 
% http://www.ctan.org/pkg/graphicx
% Another good source of documentation is "Using Imported Graphics in
% LaTeX2e" by Keith Reckdahl which can be found at:
% http://www.ctan.org/pkg/epslatex
%
% latex, and pdflatex in dvi mode, support graphics in encapsulated
% postscript (.eps) format. pdflatex in pdf mode supports graphics
% in .pdf, .jpeg, .png and .mps (metapost) formats. Users should ensure
% that all non-photo figures use a vector format (.eps, .pdf, .mps) and
% not a bitmapped formats (.jpeg, .png). The IEEE frowns on bitmapped formats
% which can result in "jaggedy"/blurry rendering of lines and letters as
% well as large increases in file sizes.
%
% You can find documentation about the pdfTeX application at:
% http://www.tug.org/applications/pdftex


% *** MATH PACKAGES ***
%
%\usepackage{amsmath}
% A popular package from the American Mathematical Society that provides
% many useful and powerful commands for dealing with mathematics.
%
% Note that the amsmath package sets \interdisplaylinepenalty to 10000
% thus preventing page breaks from occurring within multiline equations. Use:
%\interdisplaylinepenalty=2500
% after loading amsmath to restore such page breaks as IEEEtran.cls normally
% does. amsmath.sty is already installed on most LaTeX systems. The latest
% version and documentation can be obtained at:
% http://www.ctan.org/pkg/amsmath

% *** SPECIALIZED LIST PACKAGES ***
%
%\usepackage{algorithmic}
% algorithmic.sty was written by Peter Williams and Rogerio Brito.
% This package provides an algorithmic environment fo describing algorithms.
% You can use the algorithmic environment in-text or within a figure
% environment to provide for a floating algorithm. Do NOT use the algorithm
% floating environment provided by algorithm.sty (by the same authors) or
% algorithm2e.sty (by Christophe Fiorio) as the IEEE does not use dedicated
% algorithm float types and packages that provide these will not provide
% correct IEEE style captions. The latest version and documentation of
% algorithmic.sty can be obtained at:
% http://www.ctan.org/pkg/algorithms
% Also of interest may be the (relatively newer and more customizable)
% algorithmicx.sty package by Szasz Janos:
% http://www.ctan.org/pkg/algorithmicx


% *** ALIGNMENT PACKAGES ***
%
%\usepackage{array}
% Frank Mittelbach's and David Carlisle's array.sty patches and improves
% the standard LaTeX2e array and tabular environments to provide better
% appearance and additional user controls. As the default LaTeX2e table
% generation code is lacking to the point of almost being broken with
% respect to the quality of the end results, all users are strongly
% advised to use an enhanced (at the very least that provided by array.sty)
% set of table tools. array.sty is already installed on most systems. The
% latest version and documentation can be obtained at:
% http://www.ctan.org/pkg/array

% IEEEtran contains the IEEEeqnarray family of commands that can be used to
% generate multiline equations as well as matrices, tables, etc., of high
% quality.

% *** SUBFIGURE PACKAGES ***
%\ifCLASSOPTIONcompsoc
%  \usepackage[caption=false,font=footnotesize,labelfont=sf,textfont=sf]{subfig}
%\else
%  \usepackage[caption=false,font=footnotesize]{subfig}
%\fi
% subfig.sty, written by Steven Douglas Cochran, is the modern replacement
% for subfigure.sty, the latter of which is no longer maintained and is
% incompatible with some LaTeX packages including fixltx2e. However,
% subfig.sty requires and automatically loads Axel Sommerfeldt's caption.sty
% which will override IEEEtran.cls' handling of captions and this will result
% in non-IEEE style figure/table captions. To prevent this problem, be sure
% and invoke subfig.sty's "caption=false" package option (available since
% subfig.sty version 1.3, 2005/06/28) as this is will preserve IEEEtran.cls
% handling of captions.
% Note that the Computer Society format requires a sans serif font rather
% than the serif font used in traditional IEEE formatting and thus the need
% to invoke different subfig.sty package options depending on whether
% compsoc mode has been enabled.
%
% The latest version and documentation of subfig.sty can be obtained at:
% http://www.ctan.org/pkg/subfig

% *** FLOAT PACKAGES ***
%
%\usepackage{fixltx2e}
% fixltx2e, the successor to the earlier fix2col.sty, was written by
% Frank Mittelbach and David Carlisle. This package corrects a few problems
% in the LaTeX2e kernel, the most notable of which is that in current
% LaTeX2e releases, the ordering of single and double column floats is not
% guaranteed to be preserved. Thus, an unpatched LaTeX2e can allow a
% single column figure to be placed prior to an earlier double column
% figure.
% Be aware that LaTeX2e kernels dated 2015 and later have fixltx2e.sty's
% corrections already built into the system in which case a warning will
% be issued if an attempt is made to load fixltx2e.sty as it is no longer
% needed.
% The latest version and documentation can be found at:
% http://www.ctan.org/pkg/fixltx2e

%\usepackage{stfloats}
% stfloats.sty was written by Sigitas Tolusis. This package gives LaTeX2e
% the ability to do double column floats at the bottom of the page as well
% as the top. (e.g., "\begin{figure*}[!b]" is not normally possible in
% LaTeX2e). It also provides a command:
%\fnbelowfloat
% to enable the placement of footnotes below bottom floats (the standard
% LaTeX2e kernel puts them above bottom floats). This is an invasive package
% which rewrites many portions of the LaTeX2e float routines. It may not work
% with other packages that modify the LaTeX2e float routines. The latest
% version and documentation can be obtained at:
% http://www.ctan.org/pkg/stfloats
% Do not use the stfloats baselinefloat ability as the IEEE does not allow
% \baselineskip to stretch. Authors submitting work to the IEEE should note
% that the IEEE rarely uses double column equations and that authors should try
% to avoid such use. Do not be tempted to use the cuted.sty or midfloat.sty
% packages (also by Sigitas Tolusis) as the IEEE does not format its papers in
% such ways.
% Do not attempt to use stfloats with fixltx2e as they are incompatible.
% Instead, use Morten Hogholm'a dblfloatfix which combines the features
% of both fixltx2e and stfloats:
%
% \usepackage{dblfloatfix}
% The latest version can be found at:
% http://www.ctan.org/pkg/dblfloatfix

% *** PDF, URL AND HYPERLINK PACKAGES ***
%
%\usepackage{url}
% url.sty was written by Donald Arseneau. It provides better support for
% handling and breaking URLs. url.sty is already installed on most LaTeX
% systems. The latest version and documentation can be obtained at:
% http://www.ctan.org/pkg/url
% Basically, \url{my_url_here}.

% *** Do not adjust lengths that control margins, column widths, etc. ***
% *** Do not use packages that alter fonts (such as pslatex).         ***
% There should be no need to do such things with IEEEtran.cls V1.6 and later.
% (Unless specifically asked to do so by the journal or conference you plan
% to submit to, of course. )

% correct bad hyphenation here
\hyphenation{op-tical net-works semi-conduc-tor}

\usepackage{graphicx}
\graphicspath{ {./images/} }
\usepackage{hyperref}

\begin{document}
% 
% paper title
% Titles are generally capitalized except for words such as a, an, and, as,
% at, but, by, for, in, nor, of, on, or, the, to and up, which are usually
% not capitalized unless they are the first or last word of the title.
% Linebreaks \\ can be used within to get better formatting as desired.
% Do not put math or special symbols in the title.
\title{A web application access for Jobs Observer program and verification\\
{\small \today~-~\currenttime}}

 
% author names and affiliations
% use a multiple column layout for up to three different
% affiliations
\author{\IEEEauthorblockN{Stetsenko Olexandr}
\IEEEauthorblockA{University of Luxembourg\\
Email: olexandr.stetsenko.001@student.uni.lu}
\and 
\IEEEauthorblockN{Nicolas Guelfi}
\IEEEauthorblockA{University of Luxembourg\\
Email: nicolas.guelfi@uni.lu}%
}

% conference papers do not typically use \thanks and this command
% is locked out in conference mode. If really needed, such as for
% the acknowledgment of grants, issue a \IEEEoverridecommandlockouts
% after \documentclass

% for over three affiliations, or if they all won't fit within the width
% of the page (and note that there is less available width in this regard for
% compsoc conferences compared to traditional conferences), use this
% alternative format:
% 
%\author{\IEEEauthorblockN{Michael Shell\IEEEauthorrefmark{1},
%Homer Simpson\IEEEauthorrefmark{2},
%James Kirk\IEEEauthorrefmark{3}, 
%Montgomery Scott\IEEEauthorrefmark{3} and
%Eldon Tyrell\IEEEauthorrefmark{4}}
%\IEEEauthorblockA{\IEEEauthorrefmark{1}School of Electrical and Computer Engineering\\
%Georgia Institute of Technology,
%Atlanta, Georgia 30332--0250\\ Email: see http://www.michaelshell.org/contact.html}
%\IEEEauthorblockA{\IEEEauthorrefmark{2}Twentieth Century Fox, Springfield, USA\\
%Email: homer@thesimpsons.com}
%\IEEEauthorblockA{\IEEEauthorrefmark{3}Starfleet Academy, San Francisco, California 96678-2391\\
%Telephone: (800) 555--1212, Fax: (888) 555--1212}
%\IEEEauthorblockA{\IEEEauthorrefmark{4}Tyrell Inc., 123 Replicant Street, Los Angeles, California 90210--4321}}




% use for special paper notices
%\IEEEspecialpapernotice{(Invited Paper)}




% make the title area
\maketitle

% As a general rule, do not put math, special symbols or citations
% in the abstract


% no keywords

% For peer review papers, you can put extra information on the cover
% page as needed:
% \ifCLASSOPTIONpeerreview
% \begin{center} \bfseries EDICS Category: 3-BBND \end{center}
% \fi
%
% For peerreview papers, this IEEEtran command inserts a page break and
% creates the second title. It will be ignored for other modes.
\IEEEpeerreviewmaketitle

\section{Introduction}
The Bachelor Semester Project is focus to develop and to deploy a web application, called Jobs Observer, on an AWS cloud server. The web application to develop is based on an existed local software which purpose is to create statistics based on the actual jobs in Computer Science domain. 
\newline
As, the project is in the domain of Software Engineering, a small research in the field of software testing is performed.  The research consists to demonstrate the correctness of a program from the execution and behavior points of view. 

\section{Project description}
The Bachelor Semester Project is realized in different domains of Computer Science such as Software Engineering, Web development and Software testing.   7\newline             
During this project several deliverables have to be produced which can be divided into scientific and technical deliverables.

\subsubsection{Scientific deliverables}
The scientific deliverables are the demonstrations of program correctness using logical reasoning and natural deduction. There are two demonstrations in total.  
\newline
The first demonstration demonstrates the correctness of a program based on the correct execution. In other words, the result of a program has to be correct from the implementation point of view. 
\newline
The second demonstration demonstrate the correctness of a program based on requirements from the behavior point of view. 

\subsubsection{Technical deliverable}

The technical deliverable is an operational web application deployed on an AWS server. The web application, also called Jobs Observer, is composed of two parts which are the front-end and the back-end. 
\newline
The front-end is the Graphical User Interface which allows the interaction between the user and the web application.
\newline
Back-end deliverable consists to develop a web-server by using Django python framework. This web-sever is used for creating statistics based on the data of jobs from the database. It is also used for deploying the Jobs Observer on an AWS server. 


\section{Pre-requisites} 

\subsection{Natural Deduction}
Natural Deduction[1] is an intuitional method which allows to develop a proof based on hypothesis. The proof is developed based on smaller proofs which are combine at the end based on the rules for providing a justification to an assumption or hypothesis. Each proof by deduction is a sequence of lines where each line of justification is based on any previous line. 

\subsection{Unit testing}
Unit testing[2][3] is used for testing a program or some parts of the program. In unit testing the tester knows in advance what is the implementation of the program to test and can develop tests based on this knowledge. A unit test is successful when the output of the program is equal to the output of the test. 

\section{Production}

In this Bachelor Semester Project, a web application and two demonstrations (ref. 6.1 and 6.2) of correctness of a program are developed. 

\subsection{First demonstration}
The first demonstration is used for demonstrating the correctness of a program from the execution point of view. 
\newline
Firstly, a program block is declared with the inputs, outputs and a function which is the implementation of the code. A program block is a part of the program to test which can be considerate as a function. A program is composed of multiple program block. Then, one or multiple tests are created for testing this program block. 
\newline
As, the goal is to demonstrate the correctness of entire program, a set of tests is created and is composed of arrays of tests for each program block. 
\newline
At the end, the correctness of the entire program can be tested by applying the set of tests on the program. As, all the tests can be applied on each program block, a validation method is introduced. It consists to look at implementation of the program block and the test and if the both are the same then a test can be applied. For showing that the program block is successful with respect to a test, the output of the program block has to be the same as the output of the test. 
\newline
The entire program is correct if and only if all the tests are successful. 

\subsection{Second demonstration}

The second demonstration is based on the requirements which allow to show correctness of a program from the behavior point of view. Additionally, the program has to be correct from the execution point of view. 
\newline
Similarly, to the previously demonstration, a program composed of multiple program blocks and a set of tests composed of multiple arrays of tests are created. 
\newline
Then, the creation of a requirement is performed by putting together two functions which are used for validating the inputs and the outputs of a program block or of a test. Then, an array of requirement is defined for each program block to test. All the requirements are assembled together into a set which is composed of arrays of requirements. 
\newline
For showing that a requirement is satisfied, the inputs and outputs of a program block or of a test has to belongs to the requirement and the applied test has to be successful. For showing that the entire program is correct, all the requirements have to be satisfied and all the tests must be successful. 

\subsection{Web application}
The developed web application is used for creating statistics based on the jobs in Computer Science domain. This web application is composed of front-end and back-end. 
\newline
The front-end is the GUI which is used for interaction between the user and the application. It is composed of several HTML pages (ref. 6.3) which are the followed: “Home”, “About Software”, “Plots” and “Statistic”. “Home” page is used as the principal view from where other pages can be accessed. “About Software” page gives a small description about the software. “Plots” is composed of a list of possible statistics which can be choose and displayed. The last “Statistic” page is the result of a statistic selected from the “Plots” page. 
\newline
The back-end is created by using a Django framework which allows the creation of web applications.  For this web application, the back-end is used for deploying the web application on an AWS server and generating statistics when they are requested from the front-end “Plots” page. 

\section{Conclusion}
During this project, we described how the correctness of a program can be demonstrate by using the logical reasoning with the natural deduction.
Moreover, we show that it exits different types of correctness of a program, in our case execution and behavior correctness. 
\newline
Additionally, we saw some basis for developing a web application by using a Django framework and to deploy it on an AWS server. The web application developed uses a local software that generate statistics which allow to show the jobs in Computer Science.  

% An example of a floating figure using the graphicx package.
% Note that \label must occur AFTER (or within) \caption.
% For figures, \caption should occur after the \includegraphics.
% Note that IEEEtran v1.7 and later has special internal code that
% is designed to preserve the operation of \label within \caption
% even when the captionsoff option is in effect. However, because
% of issues like this, it may be the safest practice to put all your
% \label just after \caption rather than within \caption{}.
%
% Reminder: the "draftcls" or "draftclsnofoot", not "draft", class
% option should be used if it is desired that the figures are to be
% displayed while in draft mode.
%
%\begin{figure}[!t]
%\centering
%\includegraphics[width=2.5in]{myfigure}
% where an .eps filename suffix will be assumed under latex, 
% and a .pdf suffix will be assumed for pdflatex; or what has been declared
% via \DeclareGraphicsExtensions.
%\caption{Simulation results for the network.}
%\label{fig_sim}
%\end{figure}

% Note that the IEEE typically puts floats only at the top, even when this
% results in a large percentage of a column being occupied by floats.


% An example of a double column floating figure using two subfigures.
% (The subfig.sty package must be loaded for this to work.)
% The subfigure \label commands are set within each subfloat command,
% and the \label for the overall figure must come after \caption.
% \hfil is used as a separator to get equal spacing.
% Watch out that the combined width of all the subfigures on a 
% line do not exceed the text width or a line break will occur.
%
%\begin{figure*}[!t]
%\centering
%\subfloat[Case I]{\includegraphics[width=2.5in]{box}%
%\label{fig_first_case}}
%\hfil
%\subfloat[Case II]{\includegraphics[width=2.5in]{box}%
%\label{fig_second_case}}
%\caption{Simulation results for the network.}
%\label{fig_sim}
%\end{figure*}
%
% Note that often IEEE papers with subfigures do not employ subfigure
% captions (using the optional argument to \subfloat[]), but instead will
% reference/describe all of them (a), (b), etc., within the main caption.
% Be aware that for subfig.sty to generate the (a), (b), etc., subfigure
% labels, the optional argument to \subfloat must be present. If a
% subcaption is not desired, just leave its contents blank,
% e.g., \subfloat[].


% An example of a floating table. Note that, for IEEE style tables, the
% \caption command should come BEFORE the table and, given that table
% captions serve much like titles, are usually capitalized except for words
% such as a, an, and, as, at, but, by, for, in, nor, of, on, or, the, to
% and up, which are usually not capitalized unless they are the first or
% last word of the caption. Table text will default to \footnotesize as
% the IEEE normally uses this smaller font for tables.
% The \label must come after \caption as always.
%
%\begin{table}[!t]
%% increase table row spacing, adjust to taste
%\renewcommand{\arraystretch}{1.3}
% if using array.sty, it might be a good idea to tweak the value of
% \extrarowheight as needed to properly center the text within the cells
%\caption{An Example of a Table}
%\label{table_example}
%\centering
%% Some packages, such as MDW tools, offer better commands for making tables
%% than the plain LaTeX2e tabular which is used here.
%\begin{tabular}{|c||c|}
%\hline
%One & Two\\
%\hline
%Three & Four\\
%\hline
%\end{tabular}
%\end{table}


% Note that the IEEE does not put floats in the very first column
% - or typically anywhere on the first page for that matter. Also,
% in-text middle ("here") positioning is typically not used, but it
% is allowed and encouraged for Computer Society conferences (but
% not Computer Society journals). Most IEEE journals/conferences use
% top floats exclusively. 
% Note that, LaTeX2e, unlike IEEE journals/conferences, places
% footnotes above bottom floats. This can be corrected via the
% \fnbelowfloat command of the stfloats package.

% trigger a \newpage just before the given reference
% number - used to balance the columns on the last page
% adjust value as needed - may need to be readjusted if
% the document is modified later
%\IEEEtriggeratref{8}
% The "triggered" command can be changed if desired:
%\IEEEtriggercmd{\enlargethispage{-5in}}

% references section

% can use a bibliography generated by BibTeX as a .bbl file
% BibTeX documentation can be easily obtained at:
% http://mirror.ctan.org/biblio/bibtex/contrib/doc/
% The IEEEtran BibTeX style support page is at:
% http://www.michaelshell.org/tex/ieeetran/bibtex/
%\bibliographystyle{IEEEtran}
% argument is your BibTeX string definitions and bibliography database(s)
%\bibliography{IEEEabrv,../bib/paper}
%
% <OR> manually copy in the resultant .bbl file
% set second argument of \begin to the number of references
% (used to reserve space for the reference number labels box)
\begin{thebibliography}{1}

\bibitem[1]{}
\newblock {“3. Natural Deduction for Propositional Logic.” 3. Natural Deduction for Propositional Logic - Logic and Proof 0.1 Documentation,}
\newblock {https://leanprover.github.io/logic\_and\_proof/natural\_deduction\_for\_propositional\_logic.html.}

\bibitem[2]{}
\newblock {“Unit Testing.” Software Testing Fundamentals, 3 Mar. 2018,}
\newblock {http://softwaretestingfundamentals.com/unit-testing/.}

\bibitem[3]{}
\newblock {“Unit Testing Tutorial: What Is, Types, Tools, EXAMPLE.” Guru99,}
\newblock {https://www.guru99.com/unit-testing-guide.html.}

\end{thebibliography}
\newpage 
\onecolumn
\section{Appendix}

\begin{figure}[!htbp]
	\hspace{+1em}
	\includegraphics[scale=0.7]{proof-part1.png} 
	\vspace{-1.5em}
\end{figure}

\begin{figure}[!htbp]
	\includegraphics[scale=0.7]{proof-part2.png} 
\end{figure}

\begin{figure}[!htbp]
\hspace{+1em}
	\includegraphics[scale=0.7]{proof-part3.png} 
	\vspace{-9em}
\end{figure}

\begin{figure}[hbt!]
\hspace{+1em}
	\includegraphics[scale=0.7]{proof-part4.png} 
\end{figure}
\clearpage
\newpage

\textbf{6.3 Web application Front-end}
\newline\newline
\fbox{%
	\hspace{+1em}
  \parbox{50em}{The Jobs Observer web application can be found on: http://ec2-18-215-248-225.compute-1.amazonaws.com/}%
}

\begin{figure}[hbt!]
  \centering
	\includegraphics[scale=0.4]{home.png} 
	\caption{Home view of Jobs Observer web application}
\end{figure}

\begin{figure}[hbt!]
  \centering
	\includegraphics[scale=0.4]{aboutSoftware.png} 
	\caption{About Software page of Jobs Observer web application}
\end{figure}

\begin{figure}[hbt!]
  \centering
	\includegraphics[scale=0.4]{plots.png} 
	\caption{Plots page of Jobs Observer web application}
\end{figure}

\begin{figure}[hbt!]
  \centering
	\includegraphics[scale=0.4]{statistic.png} 
	\caption{A statistic based on jobs per country}
\end{figure}

\end{document}