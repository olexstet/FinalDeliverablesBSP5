\documentclass[conference,compsoc]{IEEEtran}
 
\usepackage{datetime}

% *** CITATION PACKAGES ***
%
\ifCLASSOPTIONcompsoc
  % IEEE Computer Society needs nocompress option
  % requires cite.sty v4.0 or later (November 2003)
  \usepackage[nocompress]{cite}
\else
  % normal IEEE
  \usepackage{cite}
\fi
% cite.sty was written by Donald Arseneau
% V1.6 and later of IEEEtran pre-defines the format of the cite.sty package
% \cite{} output to follow that of the IEEE. Loading the cite package will
% result in citation numbers being automatically sorted and properly
% "compressed/ranged". e.g., [1], [9], [2], [7], [5], [6] without using
% cite.sty will become [1], [2], [5]--[7], [9] using cite.sty. cite.sty's
% \cite will automatically add leading space, if needed. Use cite.sty's
% noadjust option (cite.sty V3.8 and later) if you want to turn this off
% such as if a citation ever needs to be enclosed in parenthesis.
% cite.sty is already installed on most LaTeX systems. Be sure and use
% version 5.0 (2009-03-20) and later if using hyperref.sty.
% The latest version can be obtained at:
% http://www.ctan.org/pkg/cite
% The documentation is contained in the cite.sty file itself.
%
% Note that some packages require special options to format as the Computer
% Society requires. In particular, Computer Society  papers do not use
% compressed citation ranges as is done in typical IEEE papers
% (e.g., [1]-[4]). Instead, they list every citation separately in order
% (e.g., [1], [2], [3], [4]). To get the latter we need to load the cite
% package with the nocompress option which is supported by cite.sty v4.0
% and later.

% *** GRAPHICS RELATED PACKAGES ***
%
\ifCLASSINFOpdf
  % \usepackage[pdftex]{graphicx}
  % declare the path(s) where your graphic files are
  % \graphicspath{{../pdf/}{../jpeg/}}
  % and their extensions so you won't have to specify these with
  % every instance of \includegraphics
  % \DeclareGraphicsExtensions{.pdf,.jpeg,.png}
\else
  % or other class option (dvipsone, dvipdf, if not using dvips). graphicx
  % will default to the driver specified in the system graphics.cfg if no
  % driver is specified.
  % \usepackage[dvips]{graphicx}
  % declare the path(s) where your graphic files are
  % \graphicspath{{../eps/}}
  % and their extensions so you won't have to specify these with
  % every instance of \includegraphics
  % \DeclareGraphicsExtensions{.eps}
\fi
% graphicx was written by David Carlisle and Sebastian Rahtz. It is
% required if you want graphics, photos, etc. graphicx.sty is already
% installed on most LaTeX systems. The latest version and documentation
% can be obtained at: 
% http://www.ctan.org/pkg/graphicx
% Another good source of documentation is "Using Imported Graphics in
% LaTeX2e" by Keith Reckdahl which can be found at:
% http://www.ctan.org/pkg/epslatex
%
% latex, and pdflatex in dvi mode, support graphics in encapsulated
% postscript (.eps) format. pdflatex in pdf mode supports graphics
% in .pdf, .jpeg, .png and .mps (metapost) formats. Users should ensure
% that all non-photo figures use a vector format (.eps, .pdf, .mps) and
% not a bitmapped formats (.jpeg, .png). The IEEE frowns on bitmapped formats
% which can result in "jaggedy"/blurry rendering of lines and letters as
% well as large increases in file sizes.
%
% You can find documentation about the pdfTeX application at:
% http://www.tug.org/applications/pdftex


% *** MATH PACKAGES ***
%
%\usepackage{amsmath}
% A popular package from the American Mathematical Society that provides
% many useful and powerful commands for dealing with mathematics.
%
% Note that the amsmath package sets \interdisplaylinepenalty to 10000
% thus preventing page breaks from occurring within multiline equations. Use:
%\interdisplaylinepenalty=2500
% after loading amsmath to restore such page breaks as IEEEtran.cls normally
% does. amsmath.sty is already installed on most LaTeX systems. The latest
% version and documentation can be obtained at:
% http://www.ctan.org/pkg/amsmath

% *** SPECIALIZED LIST PACKAGES ***
%
%\usepackage{algorithmic}
% algorithmic.sty was written by Peter Williams and Rogerio Brito.
% This package provides an algorithmic environment fo describing algorithms.
% You can use the algorithmic environment in-text or within a figure
% environment to provide for a floating algorithm. Do NOT use the algorithm
% floating environment provided by algorithm.sty (by the same authors) or
% algorithm2e.sty (by Christophe Fiorio) as the IEEE does not use dedicated
% algorithm float types and packages that provide these will not provide
% correct IEEE style captions. The latest version and documentation of
% algorithmic.sty can be obtained at:
% http://www.ctan.org/pkg/algorithms
% Also of interest may be the (relatively newer and more customizable)
% algorithmicx.sty package by Szasz Janos:
% http://www.ctan.org/pkg/algorithmicx


% *** ALIGNMENT PACKAGES ***
%
%\usepackage{array}
% Frank Mittelbach's and David Carlisle's array.sty patches and improves
% the standard LaTeX2e array and tabular environments to provide better
% appearance and additional user controls. As the default LaTeX2e table
% generation code is lacking to the point of almost being broken with
% respect to the quality of the end results, all users are strongly
% advised to use an enhanced (at the very least that provided by array.sty)
% set of table tools. array.sty is already installed on most systems. The
% latest version and documentation can be obtained at:
% http://www.ctan.org/pkg/array

% IEEEtran contains the IEEEeqnarray family of commands that can be used to
% generate multiline equations as well as matrices, tables, etc., of high
% quality.

% *** SUBFIGURE PACKAGES ***
%\ifCLASSOPTIONcompsoc
%  \usepackage[caption=false,font=footnotesize,labelfont=sf,textfont=sf]{subfig}
%\else
%  \usepackage[caption=false,font=footnotesize]{subfig}
%\fi
% subfig.sty, written by Steven Douglas Cochran, is the modern replacement
% for subfigure.sty, the latter of which is no longer maintained and is
% incompatible with some LaTeX packages including fixltx2e. However,
% subfig.sty requires and automatically loads Axel Sommerfeldt's caption.sty
% which will override IEEEtran.cls' handling of captions and this will result
% in non-IEEE style figure/table captions. To prevent this problem, be sure
% and invoke subfig.sty's "caption=false" package option (available since
% subfig.sty version 1.3, 2005/06/28) as this is will preserve IEEEtran.cls
% handling of captions.
% Note that the Computer Society format requires a sans serif font rather
% than the serif font used in traditional IEEE formatting and thus the need
% to invoke different subfig.sty package options depending on whether
% compsoc mode has been enabled.
%
% The latest version and documentation of subfig.sty can be obtained at:
% http://www.ctan.org/pkg/subfig

% *** FLOAT PACKAGES ***
%
%\usepackage{fixltx2e}
% fixltx2e, the successor to the earlier fix2col.sty, was written by
% Frank Mittelbach and David Carlisle. This package corrects a few problems
% in the LaTeX2e kernel, the most notable of which is that in current
% LaTeX2e releases, the ordering of single and double column floats is not
% guaranteed to be preserved. Thus, an unpatched LaTeX2e can allow a
% single column figure to be placed prior to an earlier double column
% figure.
% Be aware that LaTeX2e kernels dated 2015 and later have fixltx2e.sty's
% corrections already built into the system in which case a warning will
% be issued if an attempt is made to load fixltx2e.sty as it is no longer
% needed.
% The latest version and documentation can be found at:
% http://www.ctan.org/pkg/fixltx2e

%\usepackage{stfloats}
% stfloats.sty was written by Sigitas Tolusis. This package gives LaTeX2e
% the ability to do double column floats at the bottom of the page as well
% as the top. (e.g., "\begin{figure*}[!b]" is not normally possible in
% LaTeX2e). It also provides a command:
%\fnbelowfloat
% to enable the placement of footnotes below bottom floats (the standard
% LaTeX2e kernel puts them above bottom floats). This is an invasive package
% which rewrites many portions of the LaTeX2e float routines. It may not work
% with other packages that modify the LaTeX2e float routines. The latest
% version and documentation can be obtained at:
% http://www.ctan.org/pkg/stfloats
% Do not use the stfloats baselinefloat ability as the IEEE does not allow
% \baselineskip to stretch. Authors submitting work to the IEEE should note
% that the IEEE rarely uses double column equations and that authors should try
% to avoid such use. Do not be tempted to use the cuted.sty or midfloat.sty
% packages (also by Sigitas Tolusis) as the IEEE does not format its papers in
% such ways.
% Do not attempt to use stfloats with fixltx2e as they are incompatible.
% Instead, use Morten Hogholm'a dblfloatfix which combines the features
% of both fixltx2e and stfloats:
%
% \usepackage{dblfloatfix}
% The latest version can be found at:
% http://www.ctan.org/pkg/dblfloatfix

% *** PDF, URL AND HYPERLINK PACKAGES ***
%
%\usepackage{url}
% url.sty was written by Donald Arseneau. It provides better support for
% handling and breaking URLs. url.sty is already installed on most LaTeX
% systems. The latest version and documentation can be obtained at:
% http://www.ctan.org/pkg/url
% Basically, \url{my_url_here}.

% *** Do not adjust lengths that control margins, column widths, etc. ***
% *** Do not use packages that alter fonts (such as pslatex).         ***
% There should be no need to do such things with IEEEtran.cls V1.6 and later.
% (Unless specifically asked to do so by the journal or conference you plan
% to submit to, of course. )

% correct bad hyphenation here
\hyphenation{op-tical net-works semi-conduc-tor}

\usepackage{hyperref}

\begin{document}
% 
% paper title
% Titles are generally capitalized except for words such as a, an, and, as,
% at, but, by, for, in, nor, of, on, or, the, to and up, which are usually
% not capitalized unless they are the first or last word of the title.
% Linebreaks \\ can be used within to get better formatting as desired.
% Do not put math or special symbols in the title.
\title{Bachelor Semester Project - Web application for Computer Science Market Analysis\\
{\small \today~-~\currenttime}}

 
% author names and affiliations
% use a multiple column layout for up to three different
% affiliations
\author{\IEEEauthorblockN{Stetsenko Olexandr}
\IEEEauthorblockA{University of Luxembourg\\
Email: olexandr.stetsenko.001.student@uni.lu}
\and 
\IEEEauthorblockN{Nicolas Guelfi}
\IEEEauthorblockA{University of Luxembourg\\
Email: nicolas.guelfi@uni.lu}%
}

% conference papers do not typically use \thanks and this command
% is locked out in conference mode. If really needed, such as for
% the acknowledgment of grants, issue a \IEEEoverridecommandlockouts
% after \documentclass

% for over three affiliations, or if they all won't fit within the width
% of the page (and note that there is less available width in this regard for
% compsoc conferences compared to traditional conferences), use this
% alternative format:
% 
%\author{\IEEEauthorblockN{Michael Shell\IEEEauthorrefmark{1},
%Homer Simpson\IEEEauthorrefmark{2},
%James Kirk\IEEEauthorrefmark{3}, 
%Montgomery Scott\IEEEauthorrefmark{3} and
%Eldon Tyrell\IEEEauthorrefmark{4}}
%\IEEEauthorblockA{\IEEEauthorrefmark{1}School of Electrical and Computer Engineering\\
%Georgia Institute of Technology,
%Atlanta, Georgia 30332--0250\\ Email: see http://www.michaelshell.org/contact.html}
%\IEEEauthorblockA{\IEEEauthorrefmark{2}Twentieth Century Fox, Springfield, USA\\
%Email: homer@thesimpsons.com}
%\IEEEauthorblockA{\IEEEauthorrefmark{3}Starfleet Academy, San Francisco, California 96678-2391\\
%Telephone: (800) 555--1212, Fax: (888) 555--1212}
%\IEEEauthorblockA{\IEEEauthorrefmark{4}Tyrell Inc., 123 Replicant Street, Los Angeles, California 90210--4321}}




% use for special paper notices
%\IEEEspecialpapernotice{(Invited Paper)}




% make the title area
\maketitle

% As a general rule, do not put math, special symbols or citations
% in the abstract
\begin{abstract}
Abstract
\end{abstract}

% no keywords

% For peer review papers, you can put extra information on the cover
% page as needed:
% \ifCLASSOPTIONpeerreview
% \begin{center} \bfseries EDICS Category: 3-BBND \end{center}
% \fi
%
% For peerreview papers, this IEEEtran command inserts a page break and
% creates the second title. It will be ignored for other modes.
\IEEEpeerreviewmaketitle

\section{Introduction ($\pm$ 5\% of total words)}
% no \IEEEPARstart
Computer Science is one of the most important domains in the world and become indispensable in everyday life. As we live in a digital age, there are a lot of industries and companies which focus on improving and exploring new capabilities in this domain. Thus, this led to the grow of complexity and creation of the new sub-fields which request new skills from employees and new places in the companies and industries. 

So, a lot of new students who started to study the field of Computer Science may ask themselves, what are the possible fields of Computer Science and what are the possibilities in terms of career perspectives and the availability of jobs per field.For answering theses questions, a software was been developed for performing the actual computer science market analysis. It consists to fetch all available jobs from different countries and companies which belong to the Computer Science domain. Then, based on theses founded jobs, a lot of different statistics can be created such as tables and plots which can be used for observing the trends of the domain. Therefore, by using and observing theses statistics, students will have the possibility to choose the right field and seeing the perspectives for future career.       
                                                            
The Bachelor Semester Project will focus on deploying this recent developed software, called Jobs Observer, on a cloud server in order to make his futures available to all students, who would like to observe the actual Computer Science possibilities.                                                                                                                                   
During this project, a small but essential part of the software will be deployed and will be considerate as the basis for future upgrades. Moreover, this part must be operational and include some of possible statistics of the existing software. Then, statistics delivered by the software will be discovered for understanding at best the results for correct implementation.                                                        
As the software deployment and development are in the context of software engineering, a small research in the field of software testing will be perform. This research will consist to prove the correctness of a program by looking from the scientific point of view. 

\section{Project description ($\pm$ 10\% of total words) }
\subsection{Domains}
The Bachelor Semester Project is done in different domains of computer science such as Software Engineering, Web development and Software testing. 

\subsubsection{Scientific domain \newline\newline}

\textbf{Software testing} is a process which consists to verify and validate that a software is a bug free. The purpose of software testing is to find errors and bugs in the functionalities of a software that can lead from smaller failures to the malfunction of the overall system. Software testing is divided in verification and validation.                                                                                                                             Verification[1] process allows to understand, if the development of the software is right. It consists to verify the specified requirements in order to understand if the development of the software is conformed to these requirements.                                                                                                                           Validation[1] is a process which consists to understand, if the developed software is the right software. This process happened after the development of a software and allow to answer, if the developed software meets the requirements and the needs of the customers. 
                                              
\subsubsection{Technical domains} 

\textbf{Software Engineering} is based on the principles, which are used in the field of engineering. In Software Engineering, developers develop different types of software which can be small, large, complex or easiest. During the development, they deal with the requirements specification, software design, testing and maintenance. Moreover, developers must apply a structured and disciplined approach for being able to navigate codes, during the development and maintenance.                                   
In Software Engineering, developers must base on the customer's requirements, for providing a software that meets their needs. In addition, some other different factors have to be considerate as budget, time constraint or efficiency.

\textbf{Web development} consists to create web applications that are developed, deployed and accessed on the Internet. Web development is divided in front-end and back-end development.                                    
Front-end development consists to design the graphical interface of a web application that will allow the interaction between the system and the end users. Usually, the front-end is composed of several elements that are designed by CSS, HTML or JavaScript.                                                                                           
Back-end development is hidden part of a web application. Back-end is composed of three parts such as functionalities of software, databases and server which allows to access and perform operations on the system.

\subsection{Targeted Deliverables}
\label{sec-deliverables}
\subsubsection{Scientific deliverables}

The scientific deliverable is a presentation of program correctness demonstration using logical reasoning. The demonstration is based on natural deduction that allows to derived a conclusion based on the premises. For demonstrating the correctness of program, the demonstration is based on the principle of testing that shows if a program is correctly executed or not.  Additionally, the demonstration doesn’t only show at which conditions the program is considerate correct but why it is correct, based on idea of requirements.  

\subsubsection{Technical deliverables}
The technical deliverable is an operational software deployed on an AWS server. This technical deliverable can be divided into two parts, front-end and back-end. 
Front-end deliverable is divided into two parts. First part is a mockup of the user interface that is used for the interaction between the system and the end-users. This user interface is easy to understand, has a structure and the basic elements of a webpage.                                                               Second part is a user interface based on the mockup that is developed by using CSS and HTML languages and which is compatible with the functionalities from the back-end. 
Back-end deliverable consists to deploy partially Jobs Observer software to an AWS server. This software is developed by using Django which allows the creation of web applications. The deployed software doesn’t include all the functionalities of Jobs Observer but a part of functionalities that are operational and can be accessed through the user interface. The purpose of functionalities is to provide users with statistics based on actual Computer Science Market analysis. 



% \subsection{Constraints}
% Provide all the constraints that were to be considered for the project.
% A constraint is a property that is agreed by you and your PAT to have been satisfied before starting the project.
% An example could be ``good level in Python programming''. 
% As a consequence the work done to satisfy the constraints cannot be presented as a deliverable of the BSP.

\section{Pre-requisites ($[5\%..10\%]$ of total words)} 

\subsection{Scientific pre-requisites}
\subsubsection{Proof}
\textbf{Proof} is a logical demonstration which shows the truth of a proposition based on assumptions or axioms. For writing a proof, there are no only one method which can be used but several methods s.t direct proof, mathematical proof, natural deduction proof and more. During this project, a proof is developed based on natural deduction. 

\subsubsection{Natural deduction}
Natural deduction[2] is an intuitional method which allows to develop a proof. In Natural deduction all the developed proof is considerate as the proof by hypothesis. The proof is developed based on smaller proofs which are then combine at the end based on the rules for providing a justification to an assumption or hypothesis. Usually a proof is written as sequence of lines where for each line the justification is based on the any previous line, also called premises. 

\subsubsection{Unit testing}
Unit testing is a way of testing which purpose is to test the separate parts of a program and shows that each individual part is working correctly. For showing the correctness of execution, some requirements are defined which include the inputs and the correct outputs. In unit testing, the developer knows in advance what is the block of code to test. So, based on these knowledges, a test code will be produced for determine if a block of code to test behaves as expected. For showing that a code behaves as expected, the block to test will accept some predefined inputs from the requirements and will produce an output. Then, the test code will check if the output of the tested code is correct with respect with the output from the requirements.   

\subsection{Technical pre-requisites}
\subsubsection{Django}
\textbf{Django}[3] is open source framework written in python which allows the development of web applications. Django is composed of a set of modules that are combine together and which are used during the build for creating locally a web app. So, theses modules are the essential parts of Django because during the development, the developer will need only to configure them correctly in order to make everything work.  By using Django, the developer can develop simultaneously the frontend and back-end. For front-end, Django use templates that are HTML files which are considerate as views in views module. For back-end development, the developer is free to choose how it will be deployed but only thing is that if he wants to connect the back-end functionalities, he will need to pass them in views module. 

\subsubsection{Cloud Computing}
Cloud Computing[4,5] is used for sharing the information by using the network. The shared information can be considerate as the data which is present on a server and can be accessed through the network. A server run on real or virtual machines that are provided by the providers such Amazon. Cloud Computing gives access to servers, computer power, databases and storage that can be used for different purposes such web deployment.  Cloud Computing allows to deploy many services at the same time and give the possibility to test and understand of how many resources are needed. 

\section{ A Scientific Deliverable 1}
For each scientific deliverable targeted in section~\ref{sec-deliverables} provide a full section with all the subsections described below.
\label{sec-production}
\subsection{Requirements ($\pm$ 15\% of section's words)}
Describe here all the properties that characterize the deliverables you produced. It should describe, for each main deliverable, what are the expected functional and non functional properties of the deliverables, who are the actors exploiting the deliverables. It is expected that you have at least one scientific deliverable (e.g. ``Scientific presentation of the Python programming language'', ``State of the art on quality models for human computer interaction'', \ldots.) and one technical deliverable (e.g. ``BSProSoft - A python/django web-site for IT job offers retrieval and analysis'', \ldots). 
\subsection{Design ($\pm$ 30\% of section's words)}
Provide the necessary and most useful explanations on how those deliverables have been produced.
\subsection{Production ($\pm$ 40\% of section's words)}
Provide descriptions of the deliverables concrete production. It must present part of the deliverable (e.g. source code extracts, scientific work extracts, \ldots) to illustrate and explain its actual production.
\subsection{Assessment ($\pm$ 15\% of section's words)}
Provide any objective elements to assess that your deliverables do or do not satisfy the requirements described above. 

\section{ A Technical Deliverable 1}
For each technical deliverable targeted in section~\ref{sec-deliverables} provide a full section with all the subsections described below.
The cumulative volume of all deliverable sections represents 75\% of the paper's volume in words. Volumes below are indicated relative the the section.
\label{sec-production}
\subsection{Requirements ($\pm$ 15\% of section's words)}
cf. section~\ref{sec-production} applied to the technical deliverable
\subsection{Design ($\pm$ 30\% of section's words)}
cf. section~\ref{sec-production} applied to the technical deliverable
\subsection{Production ($\pm$ 40\% of section's words)}
cf. section~\ref{sec-production} applied to the technical deliverable
\subsection{Assessment ($\pm$ 15\% of section's words)}
cf. section~\ref{sec-production} applied to the technical deliverable


\section*{Acknowledgment}
The authors would like to thank the BiCS management and education team for the amazing work done.


\section{Conclusion}
The conclusion goes here.

% An example of a floating figure using the graphicx package.
% Note that \label must occur AFTER (or within) \caption.
% For figures, \caption should occur after the \includegraphics.
% Note that IEEEtran v1.7 and later has special internal code that
% is designed to preserve the operation of \label within \caption
% even when the captionsoff option is in effect. However, because
% of issues like this, it may be the safest practice to put all your
% \label just after \caption rather than within \caption{}.
%
% Reminder: the "draftcls" or "draftclsnofoot", not "draft", class
% option should be used if it is desired that the figures are to be
% displayed while in draft mode.
%
%\begin{figure}[!t]
%\centering
%\includegraphics[width=2.5in]{myfigure}
% where an .eps filename suffix will be assumed under latex, 
% and a .pdf suffix will be assumed for pdflatex; or what has been declared
% via \DeclareGraphicsExtensions.
%\caption{Simulation results for the network.}
%\label{fig_sim}
%\end{figure}

% Note that the IEEE typically puts floats only at the top, even when this
% results in a large percentage of a column being occupied by floats.


% An example of a double column floating figure using two subfigures.
% (The subfig.sty package must be loaded for this to work.)
% The subfigure \label commands are set within each subfloat command,
% and the \label for the overall figure must come after \caption.
% \hfil is used as a separator to get equal spacing.
% Watch out that the combined width of all the subfigures on a 
% line do not exceed the text width or a line break will occur.
%
%\begin{figure*}[!t]
%\centering
%\subfloat[Case I]{\includegraphics[width=2.5in]{box}%
%\label{fig_first_case}}
%\hfil
%\subfloat[Case II]{\includegraphics[width=2.5in]{box}%
%\label{fig_second_case}}
%\caption{Simulation results for the network.}
%\label{fig_sim}
%\end{figure*}
%
% Note that often IEEE papers with subfigures do not employ subfigure
% captions (using the optional argument to \subfloat[]), but instead will
% reference/describe all of them (a), (b), etc., within the main caption.
% Be aware that for subfig.sty to generate the (a), (b), etc., subfigure
% labels, the optional argument to \subfloat must be present. If a
% subcaption is not desired, just leave its contents blank,
% e.g., \subfloat[].


% An example of a floating table. Note that, for IEEE style tables, the
% \caption command should come BEFORE the table and, given that table
% captions serve much like titles, are usually capitalized except for words
% such as a, an, and, as, at, but, by, for, in, nor, of, on, or, the, to
% and up, which are usually not capitalized unless they are the first or
% last word of the caption. Table text will default to \footnotesize as
% the IEEE normally uses this smaller font for tables.
% The \label must come after \caption as always.
%
%\begin{table}[!t]
%% increase table row spacing, adjust to taste
%\renewcommand{\arraystretch}{1.3}
% if using array.sty, it might be a good idea to tweak the value of
% \extrarowheight as needed to properly center the text within the cells
%\caption{An Example of a Table}
%\label{table_example}
%\centering
%% Some packages, such as MDW tools, offer better commands for making tables
%% than the plain LaTeX2e tabular which is used here.
%\begin{tabular}{|c||c|}
%\hline
%One & Two\\
%\hline
%Three & Four\\
%\hline
%\end{tabular}
%\end{table}

\begin{thebibliography}{1}

\bibitem[1]{}
\newblock {Verification vs Validation.” Software Testing Fundamentals, 3 Mar. 2018}.
\newblock {http://softwaretestingfundamentals.com/verification-vs-validation/}

\bibitem[2]{}
\newblock {“3. Natural Deduction for Propositional Logic.” 3. Natural Deduction for Propositional Logic - Logic and Proof 0.1 Documentation}
https://$leanprover.github.io$/$logic_and_proof$/$natural_deduction_for_propositional_logic.html$


\bibitem[3]{}
\newblock {“What is Django?” Python For Beginners}
\newblock {https://www.pythonforbeginners.com/learn-python/what-is-django/}

\bibitem[4]{}
\newblock {“Fr.” Amazon, Sundance Pub., 2002}
\newblock {https://aws.amazon.com/fr/what-is-cloud-computing/}

\bibitem[5]{}
\newblock {Miller, Cody. “What Is Cloud Computing \& How Does 'The Cloud' Work?” Fastmetrics Business Blog, 14 Nov. 2019}
\newblock {https://www.fastmetrics.com/blog/tech/what-is-cloud-computing/}
\end{thebibliography}

% Note that the IEEE does not put floats in the very first column
% - or typically anywhere on the first page for that matter. Also,
% in-text middle ("here") positioning is typically not used, but it
% is allowed and encouraged for Computer Society conferences (but
% not Computer Society journals). Most IEEE journals/conferences use
% top floats exclusively. 
% Note that, LaTeX2e, unlike IEEE journals/conferences, places
% footnotes above bottom floats. This can be corrected via the
% \fnbelowfloat command of the stfloats package.

% trigger a \newpage just before the given reference
% number - used to balance the columns on the last page
% adjust value as needed - may need to be readjusted if
% the document is modified later
%\IEEEtriggeratref{8}
% The "triggered" command can be changed if desired:
%\IEEEtriggercmd{\enlargethispage{-5in}}

% references section

% can use a bibliography generated by BibTeX as a .bbl file
% BibTeX documentation can be easily obtained at:
% http://mirror.ctan.org/biblio/bibtex/contrib/doc/
% The IEEEtran BibTeX style support page is at:
% http://www.michaelshell.org/tex/ieeetran/bibtex/
%\bibliographystyle{IEEEtran}
% argument is your BibTeX string definitions and bibliography database(s)
%\bibliography{IEEEabrv,../bib/paper}
%
% <OR> manually copy in the resultant .bbl file
% set second argument of \begin to the number of references
% (used to reserve space for the reference number labels box)
\newpage 
\section{Appendix}
All images and additional material go there.
% that's all folks
\end{document}